\begin{frame}
\frametitle{Conclusions}

\begin{itemize}
	\item The University of Illinois is actively working to reduce GHG emissions on its campus. 
	\item A few microreactor designs would be able to produce enough hydrogen to meet MTD and UIUC fleet fuel demand.
	\item An excessive integration of PV to UIUC grid make the duck curve phenomenon likely to occur.
	\item Nuclear energy and hydrogen production proposes an approach to mitigate the negative implications of the duck curve.
	\item Hydrogen introduces a way to store energy that reduces the reliance on dispatchable sources.

\end{itemize}
\end{frame}


\begin{frame}
\frametitle{Acknowledgement}

This work is supported the NRC Faculty Development Program.
\\
I would like to thank Sam Dotson (UIUC), Beth Brunk (MTD), and Pete Varney (UIUC) for their contributions to the development of this project.

\end{frame}