\documentclass{anstrans}
%%%%%%%%%%%%%%%%%%%%%%%%%%%%%%%%%%%
\title{Validation of Spent Nuclear Fuel Output by Cyclus, a Fuel Cycle Simulator Code}
\author{Gwendolyn J. Chee, Gyutae Park, and Kathryn D. Huff}

\institute{
Dept. of Nuclear, Plasma and Radiological Engineering, University of Illinois at Urbana-Champaign \\
gchee2@illinois.edu
}

%%%% packages and definitions (optional)
\usepackage{graphicx} % allows inclusion of graphics
\usepackage{booktabs} % nice rules (thick lines) for tables
\usepackage{microtype} % improves typography for PDF
\usepackage{xspace}
\usepackage{tabularx}
\usepackage{subcaption}
\usepackage{enumitem}
\usepackage{placeins}
\usepackage[acronym,toc]{glossaries}
\newacronym[longplural={metric tons of heavy metal}]{MTHM}{MTHM}{metric ton of heavy metal}
\newacronym{ABM}{ABM}{agent-based modeling}
\newacronym{ACDIS}{ACDIS}{Program in Arms Control \& Domestic and International Security}
\newacronym{AHTR}{AHTR}{Advanced High Temperature Reactor}
\newacronym{ANDRA}{ANDRA}{Agence Nationale pour la gestion des D\'echets RAdioactifs, the French National Agency for Radioactive Waste Management}
\newacronym{APP}{APP}{Abbott Power Plant}
\newacronym{ANL}{ANL}{Argonne National Laboratory}
\newacronym{API}{API}{application programming interface}
\newacronym{ARCH}{ARCH}{autoregressive conditional heteroskedastic}
\newacronym{ARE}{ARE}{Aircraft Reactor Experiment}
\newacronym{ARFC}{ARFC}{Advanced Reactors and Fuel Cycles}
\newacronym{ARMA}{ARMA}{autoregressive moving average}
\newacronym{ASME}{ASME}{American Society of Mechanical Engineers}
\newacronym{ATWS}{ATWS}{Anticipated Transient Without Scram}
\newacronym{BDBE}{BDBE}{Beyond Design Basis Event}
\newacronym{BIDS}{BIDS}{Berkeley Institute for Data Science}
\newacronym{BOL}{BOL}{Beginning-of-Life}
\newacronym{BSD}{BSD}{Berkeley Software Distribution}
\newacronym{CAFCA}{CAFCA}{ Code for Advanced Fuel Cycles Assessment }
\newacronym{CASL}{CASL}{Consortium for Advanced Simulation of Light Water Reactors}
\newacronym{CDTN}{CDTN}{Centro de Desenvolvimento da Tecnologia Nuclear}
\newacronym{CEA}{CEA}{Commissariat \`a l'\'Energie Atomique et aux \'Energies Alternatives}
\newacronym{CI}{CI}{continuous integration}
\newacronym{CNEC}{CNEC}{Consortium for Nonproliferation Enabling Capabilities}
\newacronym{CNEN}{CNEN}{Comiss\~{a}o Nacional de Energia Nuclear}
\newacronym{CNERG}{CNERG}{Computational Nuclear Engineering Research Group}
\newacronym{COSI}{COSI}{Commelini-Sicard}
\newacronym{COTS}{COTS}{commercial, off-the-shelf}
\newacronym{CSNF}{CSNF}{commercial spent nuclear fuel}
\newacronym{CTAH}{CTAHs}{Coiled Tube Air Heaters}
\newacronym{CUBIT}{CUBIT}{CUBIT Geometry and Mesh Generation Toolkit}
\newacronym{CURIE}{CURIE}{Centralized Used Fuel Resource for Information Exchange}
\newacronym{DAG}{DAG}{directed acyclic graph}
\newacronym{DANESS}{DANESS}{Dynamic Analysis of Nuclear Energy System Strategies}
\newacronym{DBE}{DBE}{Design Basis Event}
\newacronym{DESAE}{DESAE}{Dynamic Analysis of Nuclear Energy Systems Strategies}
\newacronym{DHS}{DHS}{Department of Homeland Security}
\newacronym{DOE}{DOE}{Department of Energy}
\newacronym{DRACS}{DRACS}{Direct Reactor Auxiliary Cooling System}
\newacronym{DRE}{DRE}{dynamic resource exchange}
\newacronym{DSNF}{DSNF}{DOE spent nuclear fuel}
\newacronym{DYMOND}{DYMOND}{Dynamic Model of Nuclear Development }
\newacronym{EBS}{EBS}{Engineered Barrier System}
\newacronym{EDZ}{EDZ}{Excavation Disturbed Zone}
\newacronym{EIA}{EIA}{U.S. Energy Information Administration}
\newacronym{EPA}{EPA}{Environmental Protection Agency}
\newacronym{EP}{EP}{Engineering Physics}
\newacronym{FCO}{FCO}{Fuel Cycle Options}
\newacronym{FCT}{FCT}{Fuel Cycle Technology}
\newacronym{FCWMD}{FCWMD}{Fuel Cycle and Waste Management Division}
\newacronym{FEHM}{FEHM}{Finite Element Heat and Mass Transfer}
\newacronym{FEPs}{FEPs}{Features, Events, and Processes}
\newacronym{FHR}{FHR}{Fluoride-Salt-Cooled High-Temperature Reactor}
\newacronym{FLiBe}{FLiBe}{Fluoride-Lithium-Beryllium}
\newacronym{GCAM}{GCAM}{Global Change Assessment Model}
\newacronym{GDSE}{GDSE}{Generic Disposal System Environment}
\newacronym{GDSM}{GDSM}{Generic Disposal System Model}
\newacronym{GENIUSv1}{GENIUSv1}{Global Evaluation of Nuclear Infrastructure Utilization Scenarios, Version 1}
\newacronym{GENIUSv2}{GENIUSv2}{Global Evaluation of Nuclear Infrastructure Utilization Scenarios, Version 2}
\newacronym{GENIUS}{GENIUS}{Global Evaluation of Nuclear Infrastructure Utilization Scenarios}
\newacronym{GPAM}{GPAM}{Generic Performance Assessment Model}
\newacronym{GRSAC}{GRSAC}{Graphite Reactor Severe Accident Code}
\newacronym{GUI}{GUI}{graphical user interface}
\newacronym{HLW}{HLW}{high level waste}
\newacronym{HPC}{HPC}{high-performance computing}
\newacronym{HTC}{HTC}{high-throughput computing}
\newacronym{HTGR}{HTGR}{High Temperature Gas-Cooled Reactor}
\newacronym{IAEA}{IAEA}{International Atomic Energy Agency}
\newacronym{IEMA}{IEMA}{Illinois Emergency Mangament Agency}
\newacronym{INL}{INL}{Idaho National Laboratory}
\newacronym{IPRR1}{IRP-R1}{Instituto de Pesquisas Radioativas Reator 1}
\newacronym{IRP}{IRP}{Integrated Research Project}
\newacronym{ISFSI}{ISFSI}{Independent Spent Fuel Storage Installation}
\newacronym{ISRG}{ISRG}{Independent Student Research Group}
\newacronym{JFNK}{JFNK}{Jacobian-Free Newton Krylov}
\newacronym{LANL}{LANL}{Los Alamos National Laboratory}
\newacronym{LBNL}{LBNL}{Lawrence Berkeley National Laboratory}
\newacronym{LCOE}{LCOE}{levelized cost of electricity}
\newacronym{LDRD}{LDRD}{laboratory directed research and development}
\newacronym{LFR}{LFR}{Lead-Cooled Fast Reactor}
\newacronym{LGPL}{LGPL}{Lesser GNU Public License}
\newacronym{LLNL}{LLNL}{Lawrence Livermore National Laboratory}
\newacronym{LMFBR}{LMFBR}{Liquid-Metal-cooled Fast Breeder Reactor}
\newacronym{LOFC}{LOFC}{Loss of Forced Cooling}
\newacronym{LOHS}{LOHS}{Loss of Heat Sink}
\newacronym{LOLA}{LOLA}{Loss of Large Area}
\newacronym{LP}{LP}{linear program}
\newacronym{LWR}{LWR}{Light Water Reactor}
\newacronym{MARKAL}{MARKAL}{MARKet and ALlocation}
\newacronym{MA}{MA}{minor actinide}
\newacronym{MCNP}{MCNP}{Monte Carlo N-Particle code}
\newacronym{MILP}{MILP}{mixed-integer linear program}
\newacronym{MIT}{MIT}{the Massachusetts Institute of Technology}
\newacronym{MOAB}{MOAB}{Mesh-Oriented datABase}
\newacronym{MOOSE}{MOOSE}{Multiphysics Object-Oriented Simulation Environment}
\newacronym{MOX}{MOX}{mixed oxide}
\newacronym{MSBR}{MSBR}{Molten Salt Breeder Reactor}
\newacronym{MSRE}{MSRE}{Molten Salt Reactor Experiment}
\newacronym{MSR}{MSR}{Molten Salt Reactor}
\newacronym{NAGRA}{NAGRA}{National Cooperative for the Disposal of Radioactive Waste}
\newacronym{NCSA}{NCSA}{National Center for Supercomputing Applications}
\newacronym{NEAMS}{NEAMS}{Nuclear Engineering Advanced Modeling and Simulation}
\newacronym{NEUP}{NEUP}{Nuclear Energy University Programs}
\newacronym{NFCSim}{NFCSim}{Nuclear Fuel Cycle Simulator}
\newacronym{NFC}{NFC}{Nuclear Fuel Cycle}
\newacronym{NGNP}{NGNP}{Next Generation Nuclear Plant}
\newacronym{NMWPC}{NMWPC}{Nuclear MW Per Capita}
\newacronym{NNSA}{NNSA}{National Nuclear Security Administration}
\newacronym{NPRE}{NPRE}{Department of Nuclear, Plasma, and Radiological Engineering}
\newacronym{NQA1}{NQA-1}{Nuclear Quality Assurance - 1}
\newacronym{NRC}{NRC}{Nuclear Regulatory Commission}
\newacronym{NSF}{NSF}{National Science Foundation}
\newacronym{NSSC}{NSSC}{Nuclear Science and Security Consortium}
\newacronym{NUWASTE}{NUWASTE}{Nuclear Waste Assessment System for Technical Evaluation}
\newacronym{NWF}{NWF}{Nuclear Waste Fund}
\newacronym{NWTRB}{NWTRB}{Nuclear Waste Technical Review Board}
\newacronym{OCRWM}{OCRWM}{Office of Civilian Radioactive Waste Management}
\newacronym{ORION}{ORION}{ORION}
\newacronym{ORNL}{ORNL}{Oak Ridge National Laboratory}
\newacronym{PARCS}{PARCS}{Purdue Advanced Reactor Core Simulator}
\newacronym{PBAHTR}{PB-AHTR}{Pebble Bed Advanced High Temperature Reactor}
\newacronym{PBFHR}{PB-FHR}{Pebble-Bed Fluoride-Salt-Cooled High-Temperature Reactor}
\newacronym{PEI}{PEI}{Peak Environmental Impact}
\newacronym{PH}{PRONGHORN}{PRONGHORN}
\newacronym{PI}{PI}{Principal Investigator}
\newacronym{PNNL}{PNNL}{Pacific Northwest National Laboratory}
\newacronym{PRIS}{PRIS}{Power Reactor Information System}
\newacronym{PRKE}{PRKE}{Point Reactor Kinetics Equations}
\newacronym{PSPG}{PSPG}{Pressure-Stabilizing/Petrov-Galerkin}
\newacronym{PWAR}{PWAR}{Pratt and Whitney Aircraft Reactor}
\newacronym{PWR}{PWR}{Pressurized Water Reactor}
\newacronym{PyNE}{PyNE}{Python toolkit for Nuclear Engineering}
\newacronym{PyRK}{PyRK}{Python for Reactor Kinetics}
\newacronym{QA}{QA}{quality assurance}
\newacronym{RDD}{RD\&D}{Research Development and Demonstration}
\newacronym{RD}{R\&D}{Research and Development}
\newacronym{RELAP}{RELAP}{Reactor Excursion and Leak Analysis Program}
\newacronym{RIA}{RIA}{Reactivity Insertion Accident}
\newacronym{RIF}{RIF}{Region-Institution-Facility}
\newacronym{SAM}{SAM}{Simulation and Modeling}
\newacronym{SCF}{SCF}{Software Carpentry Foundation}
\newacronym{SFR}{SFR}{Sodium-Cooled Fast Reactor}
\newacronym{SINDAG}{SINDA{\textbackslash}G}{Systems Improved Numerical Differencing Analyzer $\backslash$ Gaski}
\newacronym{SKB}{SKB}{Svensk K\"{a}rnbr\"{a}nslehantering AB}
\newacronym{SNF}{SNF}{spent nuclear fuel}
\newacronym{SNL}{SNL}{Sandia National Laboratory}
\newacronym{SNM}{SNM}{Special Nuclear Material}
\newacronym{STC}{STC}{specific temperature change}
\newacronym{SUPG}{SUPG}{Streamline-Upwind/Petrov-Galerkin}
\newacronym{SWF}{SWF}{Separations and Waste Forms}
\newacronym{SWU}{SWU}{Separative Work Unit}
\newacronym{SandO}{S\&O}{Signatures and Observables}
\newacronym{THW}{THW}{The Hacker Within}
\newacronym{TRIGA}{TRIGA}{Training Research Isotope General Atomic}
\newacronym{TRISO}{TRISO}{Tristructural Isotropic}
\newacronym{TSM}{TSM}{Total System Model}
\newacronym{TSPA}{TSPA}{Total System Performance Assessment for the Yucca Mountain License Application}
\newacronym{UDB}{UDB}{Unified Database}
\newacronym{UFD}{UFD}{Used Fuel Disposition}
\newacronym{UML}{UML}{Unified Modeling Language}
\newacronym{UNFSTANDARDS}{UNFST\&DARDS}{Used Nuclear Fuel Storage, Transportation \& Disposal Analysis Resource and Data System}
\newacronym{UOX}{UOX}{uranium oxide}
\newacronym{UQ}{UQ}{uncertainty quantification}
\newacronym{US}{US}{United States}
\newacronym{UW}{UW}{University of Wisconsin}
\newacronym{VISION}{VISION}{the Verifiable Fuel Cycle Simulation Model}
\newacronym{VV}{V\&V}{verification and validation}
\newacronym{WIPP}{WIPP}{Waste Isolation Pilot Plant}
\newacronym{YMG}{YMG}{Young Members Group}
\newacronym{YMR}{YMR}{Yucca Mountain Repository Site}
\newacronym{NEI}{NEI}{Nuclear Energy Institute}
%\newacronym{<++>}{<++>}{<++>}
%\newacronym{<++>}{<++>}{<++>}

\makeglossaries
\newcommand{\SN}{S$_N$}
\renewcommand{\vec}[1]{\bm{#1}} %vector is bold italic
\newcommand{\vd}{\bm{\cdot}} % slightly bold vector dot
\newcommand{\grad}{\vec{\nabla}} % gradient
\newcommand{\ud}{\mathop{}\!\mathrm{d}} % upright derivative symbol
\newcommand{\Cyclus}{\textsc{Cyclus}\xspace}%
\newcommand{\Cycamore}{\textsc{Cycamore}\xspace}%
\newcolumntype{c}{>{\hsize=.56\hsize}X}
\newcolumntype{b}{>{\hsize=.7\hsize}X}
\newcolumntype{s}{>{\hsize=.74\hsize}X}
\newcolumntype{f}{>{\hsize=.1\hsize}X}
\newcolumntype{a}{>{\hsize=.45\hsize}X}
\usepackage{titlesec}
\titleformat*{\subsection}{\normalfont}

\begin{document}
%%%%%%%%%%%%%%%%%%%%%%%%%%%%%%%%%%%%%%%%%%%%%%%%%%%%%%%%%%%%%%%%%%%%%%%%%%%%%%%%
\section{Introduction}
\Cyclus \cite{carlsen_cyclus_2014}, a nuclear fuel cycle simulator, was used to simulate the
United States' nuclear fuel cycle from 1968 through 2013. The
\gls{SNF} inventory from the \Cyclus simulation was compared to the \gls{SNF} 
inventory from the \gls{DOE} sponsored \gls{UNFSTANDARDS} \gls{UDB} 
\cite{peterson_unf-st&dards_2017}. The \gls{UDB} provides comprehensive and 
consistent technical data on reactor sites and \gls{SNF} from the beginning of 
nuclear reactor operation in the \gls{US} until 2013. This comparison 
between \Cyclus and \gls{UDB} establishes a realistic validation of \Cyclus' 
capability to produce spent fuel mass and isotopic compositions 
that closely match reality.

%%%%%%%%%%%%%%%%%%%%%%%%%%%%%%%%%%%%%%%%%%%%%%%%%%%%%%%%%%%%%%%%%%%%%%%%%%%%%%%%
\section{Motivation}
The \gls{US} is currently considering various geologic disposal 
options \cite{DOE_strategy_2013}. Decisions such as waste package spacing, 
waste repository geometry and size will be influenced by key criteria such as 
thermal load of waste packages and the thermal capacity of the selected 
geologic host media. Waste package thermal evolution depends on the decay heat 
contribution from each isotope in the spent fuel. Therefore, to correctly 
simulate loading of a nuclear waste repository based on thermal constraints in \Cyclus, 
the simulation must first give isotopic compositions and spent fuel masses that 
closely replicate reality. 

%%% 
\section{Background}
\Cyclus is an agent-based nuclear fuel cycle simulation framework 
\cite{huff_fundamental_2016}, which means that each entity (i.e. Region, 
Institution, or Facility) in the fuel cycle is modeled as an agent. 
\Cycamore \cite{carlsen_cycamore_2014} provides agents to represent process 
physics of various components of the nuclear fuel cycle (e.g. mine, fuel 
enrichment facility, reactor) \cite{huff_extensions_2014}. The nuclear fuel 
assemblies tracked within the simulation are recipe-based; recipes specify mass 
fractions of each isotope for fresh and spent fuel. They are calculated ahead 
of time using neutronics depletion analysis tools such as ORIGEN 
\cite{bell_origen_1973}, then entered directly into the fuel cycle simulation 
\cite{peterson_additional_2017}. 

%%%%%%%%%%%%%%%%%%%%%%%%%%%%%%%%%%%%%%%%%%%%%%%%%%%%%%%%%%%%%%%%%%%%%%%%%%%%%%%%
\section{Methodology}
Published data of the 112 commercial nuclear reactors that have operated since 
1968 was used to create a \Cyclus simulation of the \gls{US} 
nuclear fuel cycle. The reactor deployment data was obtained from the \gls{PRIS} 
database \cite{IAEA_pris_2017}.  Relevant data includes 
reactor unit, reactor type, net capacity (MWe), first grid date, and shutdown 
date. \gls{US}' reactor data was extracted and used to populate the \Cyclus 
simulation. The recipes used in the \Cyclus simulations are taken from a 
reference depletion calculation done using ORIGEN \cite{bell_origen_1973} for 
burnups of 51 and 33 GWD/MTU. They were also used in 
\cite{wilson_adoption_2009, bae_synergistic_2017}. 

Jinja2 \cite{ronacher_welcome_2018}, a Python templating language, was then 
used to render the data into an input file that is accepted by 
\Cyclus. The output file produced by \Cyclus was also analyzed using Python. 

The assumptions made for this \Cyclus simulation include: 

\begin{enumerate}[topsep=0pt,itemsep=-1ex,partopsep=1ex,parsep=1ex]
	\item Cycle time is 18 months. 
	\item Refueling time is 1 month. 
	\item There is isotopic decay. 
\end{enumerate}

The \gls{UDB} database contains commercial \gls{SNF} information from 1968 through 2013. 
Data such as initial enrichment, burnup, mass of spent fuel and discharged 
dates were collected from multiple sources \cite{peterson_additional_2017}. 
Irradiation and decay calculations were performed on each spent fuel assembly based on their parameters
in the collected data \cite{peterson_additional_2017}.

The \gls{UDB} dataset used for this work included discharged fuel assembly data per 
reactor, specific isotopic concentrations and decay heat for each assembly 
along with its discharge date \cite{peterson_unf-st&dards_2017}. The \gls{UDB} 
dataset was processed and compared with the \Cyclus 
simulations' outputs. All scripts and data used are available in a public github repository
\cite{chee_arfc/transition-scenarios_2018}. 
%%%%%%%%%%%%%%%%%%%%%%%%%%%%%%%%%%%%%%%%%%%%%%%%%%%%%%%%%%%%%%%%%%%%%%%%%%%%%%%%
\section{Results and Analysis}
The primary goal of this study is to provide comparisons between 
\Cyclus data and \gls{UDB} data for spent fuel mass and isotopic contributions to the 
spent fuel mass. 

\subsection{\textit{Cumulative Spent Fuel Mass Comparison}}
To investigate the accuracy of \Cyclus, a comparison between the cumulative spent fuel mass 
predicted by \Cyclus and calculated in \gls{UDB} was conducted. As seen in figure \ref{fig:total_original}, the cumulative spent fuel mass estimated by \Cyclus is larger than the \gls{UDB} calculation before the year 2000; after 2000, the \gls{UDB} calculation reports a larger cumulative spent fuel mass. The discrepancies can be attributed to the rigidity of the \Cyclus simulation input with respect to cycle and 
refueling durations. In \Cyclus, the user specifies refueling and cycle times for each 
reactor as constant integer months.  In reality, the cycle and refueling durations 
vary throughout each reactor's lifetime and are not exact integer multiples of one month. 

\begin{figure}[t] % replace 't' with 'b' to force it to be on the bottom
	\centering
	\includegraphics[width=0.48\textwidth]{../figures/cumulative_mass_udb_cyclus}
	\caption{The cumulative spent fuel mass against discharge time for \Cyclus and \gls{UDB} data from 1968 through 2013.}
	\label{fig:total_original}
\end{figure}

\begin{figure}[t] 
	\centering
	\includegraphics[width=0.48\textwidth]{../figures/cumulative_mass_udb_cyclus_refueltime}
	\caption{The cumulative spent fuel mass against discharge time for \Cyclus and \gls{UDB} data from 1968 through 2013 for various refueling durations.}
	\label{fig:total_refueltime}
\end{figure} 

The \gls{NEI} reported significant variance in the refueling duration for \gls{US} reactors. Average 
refueling duration in 1990 was 104 days and decreased to an average of 35 days in 2017 \cite{iaea_current_nodate}. In the \Cyclus simulations, a constant refueling duration of 1 month was 
used. Therefore, to investigate if the refueling duration could explain the spent fuel mass discrepancy in figure \ref{fig:total_original}, \Cyclus simulations were run with different refueling durations.
Figure \ref{fig:total_refueltime} shows the cumulative spent fuel mass from \Cyclus simulations where 
the refueling duration was varied. A longer refueling duration brings the cumulative spent fuel mass 
from \Cyclus simulations closer to the \gls{UDB} data before 2000.

\begin{figure}[b] % replace 't' with 'b' to force it to be on the bottom
	\centering
	\includegraphics[width=0.48\textwidth]{../figures/cumulative_mass_udb_cyclus_cycletime}
	\caption{The cumulative spent fuel mass against discharge time for \Cyclus and \gls{UDB} data from 1968 to 2013 for various cycle times.}
	\label{fig:total_cycletime}
\end{figure} 

The larger cumulative \gls{UDB} spent fuel mass compared to the \Cyclus simulation 
after 2000 can be attributed to real world cycle lengths being shorter on 
average than the 18 month cycle time assumed in the \Cyclus simulations. 
Figure \ref{fig:total_cycletime} shows the cumulative spent fuel mass from 
\Cyclus simulations where cycle duration is varied. A shorter cycle duration brings the 
cumulative spent fuel mass from \Cyclus simulations closer to the \gls{UDB} data after 
2000. The \gls{US} \gls{DOE} reported a downward trend of forced outage 
rates for nuclear reactors from 4.24\%  in 2000 to 2.98\% in 2014 \cite{gehin_nuclear_2016}. 
The decreased rate of forced outages from 2000 to 2013 implies that the cycle duration also 
decreased. 

\subsection{\textit{Major Isotopic Composition of  Spent Fuel Mass Comparison}}
To accurately simulate the U.S. nuclear fuel cycle from 1968 through the present, 
it is important to use spent fuel recipes that have similar burnup in relation to 
the average U.S. nuclear reactor burnup. 

Figure \ref{fig:burn_up_real} shows the average cumulative burnup for U.S. 
nuclear reactors from 1968 to 2013 \cite{eia_spent_2015}. Figure 
\ref{fig:burn_up_difference} shows the difference between burnup of the spent 
fuel recipes used in the \Cyclus simulations and actual cumulative burnup of U.S. 
nuclear reactors as seen in figure \ref{fig:burn_up_real}. On average, spent 
fuel burnup of 33 GWD/MTU is closer to the actual cumulative burnup of U.S. nuclear 
reactors than 51 GWD/MTU. 

\begin{figure}[t] % replace 't' with 'b' to force it to be on the bottom
	\centering
	\includegraphics[width=0.48\textwidth]{../figures/burn_up_real}
	\caption{The average cumulative burnup for U.S. nuclear reactors from 1968 to 2013 \cite{eia_spent_2015}.}
	\label{fig:burn_up_real}
\end{figure} 

\begin{figure}[t] % replace 't' with 'b' to force it to be on the bottom
	\centering
	\includegraphics[width=0.48\textwidth]{../figures/burn_up_difference}
	\caption{The difference between average cumulative burnup for U.S. nuclear reactors and burnup used in \Cyclus simulations.}
	\label{fig:burn_up_difference}
\end{figure} 

Figures \ref{fig:absolute_diff_all_51} and \ref{fig:absolute_diff_all_33} show 
the cumulative spent fuel isotopic mass difference between \gls{UDB} and \Cyclus data 
in 5 year intervals for burnup rates of 51 GWD/MTU and 33 GWD/MTU correspondingly. 
The \Cyclus data that had 33 GWD/MTU burnup deviated less compared to the 
\Cyclus data that had 51 GWD/MTU burnup. This is apparent for $^{236}$U, 
$^{242}$Pu and $^{240}$Pu. However, they are similar for the isotopes on the left side 
of both figures, with an exception of $^{239}$Pu that deviated more for the \Cyclus data that 
had 33 GWD/MTU burnup. 

\begin{figure*}[htb] % replace 't' with 'b' to force it to be on the bottom
	\centering
        \begin{subfigure}{0.9\textwidth}
        \includegraphics[height=0.30\textheight]{../figures/absolute_diff_all_51}
        \caption{51 GWD/MTU burnup.}
	\label{fig:absolute_diff_all_51}
        \end{subfigure}
        \begin{subfigure}{0.9\textwidth}
        \includegraphics[height=0.30\textheight]{../figures/absolute_diff_all_33}
        \caption{33 GWD/MTU burnup.}
	\label{fig:absolute_diff_all_33}
        \end{subfigure}
        \begin{subfigure}{0.9\textwidth}
        \includegraphics[height=0.35\textheight]{../figures/absolute_diff_2000}
        \caption{Both burnup states, year 2000 data.}
	\label{fig:absolute_diff_2000}
        \end{subfigure}
        \caption{The absolute difference between cumulative spent fuel mass calculated by 
        \gls{UDB} and \Cyclus for each isotope. Positive difference indicates \Cyclus 
        mass estimate is larger.}
        \label{fig:absolute}
\end{figure*} 
\FloatBarrier

To demonstrate the impact of burnup on isotopic composition, the cumulative 
spent fuel isotopic mass difference between \gls{UDB} and \Cyclus data for 
the year 2000 is shown in figure \ref{fig:absolute_diff_2000}. In 2000, the difference in 
burnup between the actual U.S. reactor burnups and \Cyclus data burnup was 
approximately 20 GWD/MTU for 51 GWD/MTU burnup and 0 GWD/MTU for 33 GWD/MTU burnup 
(figure \ref{fig:burn_up_difference}), therefore, the impact of burnup on isotopic 
composition can be seen more clearly since one \Cyclus simulation has the same burnup as the 
actual U.S nuclear reactors burnup. 

Wigeland et al \cite{wigeland_separations_2006} discussed that $^{240}$Pu, 
$^{239}$Pu and $^{241}$Am are the most significant long-term decay heat 
contributors to each waste package, while, $^{238}$Pu, $^{244}$Cm, $^{90}$Sr 
and $^{137}$Cs are the most significant short term decay heat contributors 
\cite{wigeland_separations_2006}. 

In figure \ref{fig:absolute_diff_2000}, the \Cyclus simulation that uses the 33 
GWD/MTU burnup recipe has a small mass difference between \gls{UDB} and \Cyclus data 
for $^{240}$Pu and $^{241}$Am compared to 51 GWD/MTU burnup. However, it has a 
substantial difference for $^{239}$Pu. Figure 4 also shows small differences 
between \gls{UDB} and \Cyclus data for $^{238}$Pu, $^{244}$Cm, $^{90}$Sr and 
$^{137}$Cs. 

The large mass difference between \gls{UDB} and \Cyclus data (where \gls{UDB} $^{239}$Pu 
mass is larger than \Cyclus $^{239}$Pu mass) for $^{239}$Pu in figures 
\ref{fig:absolute_diff_2000}, \ref{fig:absolute_diff_all_51} and 
\ref{fig:absolute_diff_all_33} can be attributed to conservative depletion 
parameters used in the calculations for isotopic compositions in the \gls{UDB} 
database \cite{peterson_additional_2017}. These assumptions result in the 
hardening of the neutron spectrum that results in increased $^{239}$Pu 
production in the \gls{UDB} data \cite{peterson_additional_2017}. 

%%%%%%%%%%%%%%%%%%%%%%%%%%%%%%%%%%%%%%%%%%%%%%%%%%%%%%%%%%%%%%%%%%%%%%%%%%%%%%%%
\section{Conclusions}
This work demonstrates that the spent fuel mass and isotopic composition 
calculated by the \Cyclus simulation for the \gls{US} nuclear fuel cycle 
follow similar trends as the real world metrics. Deviations from the real world metric 
can be explained by issues with the \Cyclus model, such as the reactor agent's rigidity 
of only accepting integer month values for cycle and refueling durations, and a single 
spent fuel recipe. To replicate reality more closely, future work 
will give the reactor agent the capability to accept varying cycle lengths, refueling 
durations and spent fuel recipes.

%%%%%%%%%%%%%%%%%%%%%%%%%%%%%%%%%%%%%%%%%%%%%%%%%%%%%%%%%%%%%%%%%%%%%%%%%%%%%%%%
\section{Acknowledgments}
This research is funded by the \gls{DOE} Office of 
Nuclear Energy's Nuclear Energy University Program (Project 16-10512) 
"Demand-Driven Cycamore Archetypes". The authors want to thank members of the 
\gls{ARFC} group at the University of Illinois at Urbana-Champaign, 
particularly Jin Whan Bae and Gregory Westphal. We also thank our colleagues 
from the \Cyclus community, particularly those in the University of Wisconsin 
\gls{CNERG} and the University of South Carolina Energy Research Group (ERGS) 
for collaborative \Cyclus development.

%%%%%%%%%%%%%%%%%%%%%%%%%%%%%%%%%%%%%%%%%%%%%%%%%%%%%%%%%%%%%%%%%%%%%%%%%%%%%%%%
\bibliographystyle{ans}
\bibliography{bibliography}
\end{document}

