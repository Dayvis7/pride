\section{Methodology}

In this work we use and extend the methodology described in Baker et. al using the \texttt{RAVEN} framework \cite{baker_optimal_2018,alfonsi_raven_2016}. This methodology allows for empirical quantification of energy system flexibility. Baker et. al used a fixed amount of nuclear power production from a small modular reactor (300 MWe) and varied the penetration of VRE in the form of wind power. They also included grid flexibility in the form of a fixed amount of battery storage. UIUC has wind and solar power purchase agreements that fix the amount of VRE the campus receives. Rather than fixing the nuclear power production, this methodology will be extended to search for the optimal size of a micro-reactor for the campus. \\
There are three main steps to this methodology \cite{baker_optimal_2018}: 

\begin{enumerate}
	\item Generate synthetic time histories by training a reduced order model (ROM) with historical data using the \texttt{TrainARMA} functionality in \texttt{RAVEN}. Historical data is condensed into a typical time history which attempts to remove anomolous data.
	\item Calculate the net demand for each history set.
	\item Pass the net demand to the nuclear hybrid energy system (NHES) dispatch model defined for the UIUC embedded grid. 
\end{enumerate}

\subsection{Net Demand}
Similar to Baker et. al, we define the net demand to be 


\begin{equation}
	\begin{split}
		netDemand_h & = demand_h - (windPower_h + solarPower_h) \\ 
		& \text{$\forall$ $h$ in } [0,8759]
	\end{split}
\end{equation}

Where $demand_h$ is the synthetic time history from the ARMA model. This is because VRE is used regardless of demand (thus it could exceed demand). $windPower_h$ is computed by \cite{garcia_nuclear_2015}:

\begin{equation}
	windPower_h = \begin{cases}
		0 &\text{ if $U_h \geq 25$}\\
		0.5\eta\rho U_h^3\frac{\pi d^2}{4} &\text{ $3 < U_h \leq 12$}\\
		8.6 &\text{ $12 < U_h < 25$}
	\end{cases}
\end{equation}

\begin{equation*}
	\text{where} \begin{cases}
			\eta &\text{is the conversion efficiency (0.31)}\\
	\rho &\text{is the density of air at the site}\\
	U &\text{is the wind speed in [m/s]}\\
	d &\text{is the diameter of the turbine blades (77 m)}
	\end{cases}
\end{equation*}

The value for peak output is based on the UIUC wind power purchase agreement \cite{breitweiser_wind_2016}. $solarPower_h$ is available as historical data from AlsoEnergy \cite{alsoenergy_university_2019}. There are times when the meters failed to record data so power output for those times was calculated by \cite{garcia_nuclear_2015}:

\begin{equation}
 	P = G_T\tau_{pv}\eta_{ref}A[1-\gamma(T-25)]
\end{equation} 
where 
\begin{equation}
	G_T = DNI*\cos(\beta+\delta-lat)+DHI*\frac{180-\beta}{180}
\end{equation}
and
\begin{equation}
	\delta = 23.44*\sin(\frac{\pi}{180}\frac{360}{365}(N+284))
\end{equation}

\begin{equation*}
	\text{where } \begin{cases}
	N &\text{is the day of the year}\\
	DNI &\text{is the Direct Normal Irradiance [kW]}\\
	DHI &\text{is the Diffuse Horizontal Irradiance [kW]}\\
	G_T &\text{is total irradiance [kW]}\\
	\eta &\text{is the conversion efficiency (0.15)}\\
	\beta &\text{is the tilt of the solar panels [$degrees$]}\\
	\delta &\text{is the declination of the Earth [$degrees$]}\\
	T &\text{is the temperature [$^\circ$C]}\\
	A &\text{is the area covered by solar farm [$m^2$]} \\
	\gamma &\text{is the temperature coefficient (0.0045)}\\
	\tau &\text{is the transmittance of the PV module}
	\end{cases}
\end{equation*}
These values were obtained from the iSEE facts sheet about the UIUC solar farm \cite{white_solar_2017}. 

\subsection{Dispatch Model}

The dispatch model is similar to that used by Baker et. al \cite{baker_optimal_2018}.

% \begin{figure}
% 	\begin{center}
% 		\begin{tikzpicture}[node distance = 1.5cm, auto]
% 			\node [elli, ]
% 		\end{tikzpicture}
% 	\end{center}
% \end{figure}