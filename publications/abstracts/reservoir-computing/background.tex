\section{Background}

Reservoir computing (RC) is a modern machine learning algorithm that enables
accurate short
to medium term predictions. Ott et. al demonstrated that reservoir computing
can be used to accurately predict the evolution of a chaotic system up to seven
Lyapunov times in the
future \cite{pathak_model-free_2018, wikner_combining_2020}. A Lyapunov time is
the timescale on which a dynamic system expresses chaos
due to small deviations in initial conditions. Conventional predictions succumb
to chaos after one
Lyapunov time and thus become useless. The Lyapunov time for a
weather system is on the order of a few days but depends on regional
environment.
Electricity production from solar and wind are tightly coupled to regional
weather. Electricity demand exhibits some seasonal regularity but is still
subject to stochasticity. Combining accurate demand predictions with reliable
renewable energy predictions will enable an artificially intelligent reactor
operator to adjust power in a relaxed manner. Additionally, reservoir computing
is relatively computationally inexpensive and fast to train. This is owed to its
sparse network architecture \cite{pathak_model-free_2018,
wikner_combining_2020, vannitsem_predictability_2017}.
