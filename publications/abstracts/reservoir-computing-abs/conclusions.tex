\section{Conclusions}

The ability of an \gls{ESN} to predict dramatic changes in electricity demand
may cause skepticism, especially given the simplicity of its implementation.
It is important to note that a chaotic spatiotemporal system, like weather, is
the result of complex dynamics and not randomness. We demonstrated that even a
basic \gls{ESN} can predict the evolution of a dynamic system, such as
electricity demand.
More work needs to be done to improve prediction accuracy for the purpose of
load following with nuclear power plants. In the future we will use \glspl{ESN}
to forecast electricity production from solar and wind. We will also explore the
impact of training length and network size on accuracy. Finally, we must
determine an optimum set of input values for net demand prediction. These values
might include temperature, wind speed, air pressure, and irradiance because of
possible interactions among this data that could influence electricity
production from renewable sources.
