\section{Introduction}

Many nuclear power plants are at risk of shutting down due to a combination of
policy decisions and financial strain \cite{clemmer_nuclear_2018}. Accurately
predicting the generation from solar and wind may help minimize profit losses 
for nuclear power plants. Nuclear
power competes with natural gas to fulfill the net demand after accounting for
renewable energy sources. Load following gives natural
gas an economic edge over nuclear power because natural gas plants can follow
grid demand and even shut off when renewable penetration makes the
price of electricity go negative \cite{keppler_carbon_2011}. It is possible for
existing nuclear plants to load follow somewhat and some French nuclear
plants have been retrofitted to follow daily variations in electricity demand
\cite{lokhov_technical_2011}. As penetration of solar
and wind energy increases around the world, load following will depend less on
electric demand profiles and more on renewable energy profiles. This shift makes
load following with current nuclear power plants intractable
\cite{cany_nuclear_2018}.
Renewable energy challenges the base load electricity production that
nuclear provides in the United States by introducing grid demand variability.
Advanced reactor designs, like some \glspl{MSR}, promise strong load following
capabilities due to active $^{135}$Xe removal \cite{rykhlevskii_impact_2019}.
Unfortunately, the most mature MSR
designs are at least a decade away from obtaining a commercial license in the
United States. The climate crisis is too urgent to wait this long for nuclear
power to become fully competitive with natural gas
\cite{intergovernmental_panel_on_climate_change_climate_2014}.

Variability has been the primary drawback for renewable energy sources like
wind turbines, solar PV, and solar concentrators, since their inception. This
flaw worsens as renewable energy penetration increases
\cite{cany_nuclear_2018}. Forecasting electricity production from
renewable sources is therefore important for successful management of power
systems \cite{kobylinski_high-resolution_2020}. Recent studies applied
\glspl{ann}, specifically multi-layer perceptrons, to the task of net load
forecasting \cite{kobylinski_high-resolution_2020,dutta_load_2017,lee_development_2016}.
These studies made short term forecasts of 4-6 hours but nuclear plants
need accurate forecasts further ahead to facilitate load following.
This study will be the first to apply \glspl{ESN} to the task of net load
prediction.
