\section{Introduction}

Load following has given natural gas an economic edge over nuclear power
because natural gas plants can follow grid demand and even shut off when
renewable penetration makes the
price of electricity go negative \cite{keppler_carbon_2011}. It is possible for
existing nuclear plants to load follow somewhat. Some French nuclear
plants have been retrofitted to follow daily
variations in electricity demand \cite{lokhov_technical_2011}. Unlike those in
the United States, French nuclear power plants enjoy a majority share of the
country's electric generation. The high degree of nuclear energy penetration
makes electricity demand predictable and enables synchronized operation.
Renewable energy challenges the base load electricity production that
nuclear provides in the United States by introducing grid demand variability
that is much less predictable. As penetration of solar
and wind energy increases around the world, load following will depend less on
electric demand profiles and more on renewable energy profiles. This shift makes
load following with current nuclear power plants intractable \cite{cany_nuclear_2018}.
Advanced reactor designs, like some \glspl{MSR}, promise strong load following
capabilities due to harder
neutron spectra and active $^{135}$Xe removal \cite{rykhlevskii_impact_2019}.

Unfortunately, the most mature MSR
designs are at least a decade away from obtaining a commercial license in the
United States. The climate crisis is too urgent to wait this long for nuclear
power to become fully competitive with natural gas.
Variability has been the primary drawback for renewable energy sources like
wind turbines, solar PV, and solar concentrators, since their inception. This
flaw has become more pronounced as renewable penetration on the electricity
grid increased in recent years. Forecasting electricity production from
renewable sources is therefore important for successful management of power
systems \cite{kobylinski_high-resolution_2020}. Recent studies have applied
\glspl{ann}, specifically multi-layer perceptrons, to the task of net load
forecasting \cite{kobylinski_high-resolution_2020,dutta_load_2017,lee_development_2016}.
These studies made short term forecasts of 4-6 hours but nuclear plants
need accurate forecasts further ahead to facilitate load following.
This study will be the first to apply \glspl{ESN} to the task of net load
prediction.
