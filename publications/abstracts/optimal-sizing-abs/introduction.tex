\section{Introduction}

There has been significant work recently to develop energy system concepts that incorporate a mixture of nuclear energy, variable renewable energy (VRE) and energy storage techniques.
These systems are often referred to as nuclear hybrid energy systems (NHES) and are generally robust, reliable, economically appealing, and have low to zero greenhouse gas (GHG) emissions \cite{baker_optimal_2018,ruth_nuclear-renewable_2014,ruth_economic_2016,suman_hybrid_2018-1}.
In 2015, the University of Illinois at Urbana-Champaign (UIUC) made a commitment to reach carbon neutrality by 2050 in the Illinois Climate Action Plan (iCAP) \cite{isee_illinois_2015}.
This plan identified nuclear energy as a potential contributor to this goal.
In this work, we use the \texttt{RAVEN} framework to find the optimal size for a micro-reactor that would minimize the levelized cost of electricity (LCOE) for the UIUC embedded grid.
The greatest source of economic variability is contained in the capital costs of a micro-reactor. Thus we examine both scenarios where the reactor is provided at no cost to the university and purchased at full price.
