\section{Methodology}

In this work we use and extend the methodology described in Baker et. al using
the \texttt{RAVEN}
framework \cite{baker_optimal_2018,alfonsi_raven_2016}.
This methodology allows for empirical quantification of energy system
flexibility.
Baker et. al used a fixed amount of nuclear power production from a small
modular reactor (300
MWe) and varied the penetration of VRE in the form of wind power.
They also included grid flexibility in the form of a fixed amount of battery
storage. UIUC has
wind and solar power purchase agreements that fix the amount of VRE the campus
receives.
Rather than fixing the nuclear power production, this methodology will be
extended to search for
the optimal size of a micro-reactor for the campus.
Additionally, the reactor will serve as baseload for steam demand with excess
thermal power
diverted to generate electricity.\\
There are three main steps to this methodology \cite{baker_optimal_2018}:

\begin{enumerate}
	\item Generate synthetic data by training a reduced order model (ROM) with
	historical data using
	\texttt{RAVEN}'s \texttt{TrainARMA} functionality. \texttt{RAVEN} condenses
	historical data into
	a typical time series and attempts to remove anomalous data.
	\item Calculate the net demand for each history set.
	\item Pass the net demand to the nuclear hybrid energy system (NHES) dispatch
	model defined for
	the UIUC embedded grid.
\end{enumerate}

\subsection{Net Demand}
Similar to Baker et. al, we define the net demand to be


\begin{equation}
	\label{eqn:net-demand}
	\begin{split}
		D_{net} & = (D_{total} - P_{wind} - P_{pv})_h \text{ $\forall$ $h$ in }
		[0,8759].
	\end{split}
\end{equation}

Where $D_{total}$ is the total demand at a given hour, $h$, of a synthetic
demand profile from the
ARMA model. We adjust the net demand, $D_{net}$, because VRE is used regardless
of demand (thus it
could exceed demand). $P_{wind}$ is computed by \cite{garcia_nuclear_2015}:

\begin{align}
	P_{wind} &= \begin{cases}
		0 &\text{ if $U_h \geq 25$}\\
		0.5\eta\rho U_h^3\frac{\pi d^2}{4} &\text{ $3 < U_h \leq 12$}\\
		8.6 &\text{ $12 < U_h < 25$}\\
	\end{cases}
\end{align}

\begin{align*}
	\intertext{where}
	\eta &\text{ is the conversion efficiency (0.31)}\\
	\rho &\text{ is the density of air at the site}\\
	U &\text{ is the wind speed in [m/s]}\\
	d &\text{ is the diameter of the turbine blades (77 m)}
\end{align*}

The value for peak output is based on the UIUC wind power purchase agreement
\cite{breitweiser_wind_2016}. $P_{pv}$ is available as historical data from
AlsoEnergy
\cite{alsoenergy_university_2019}. There are times when the meters failed to
record data so power
output for those times was calculated by \cite{garcia_nuclear_2015}:

\begin{align}
 	P_{pv} &= G_T\tau_{pv}\eta_{ref}A[1-\gamma(T-25)]\\
\intertext{where}
	G_T &= DNI*\cos(\beta+\delta-lat)+DHI*\frac{180-\beta}{180}\\
	\delta &= 23.44*\sin\left(\frac{\pi}{180}\frac{360}{365}(N+284)\right)
	\intertext{where }
	N &=\mbox{ day of the year}\nonumber\\
	DNI &=\mbox{ Direct Normal Irradiance [kW]}\nonumber\\
	DHI &=\mbox{ Diffuse Horizontal Irradiance [kW]}\nonumber\\
	G_T &=\mbox{ total irradiance [kW]}\nonumber\\
	\eta &=\mbox{ conversion efficiency (0.15)}\nonumber\\
	\beta &=\mbox{ tilt of the solar panels [$degrees$]}\nonumber\\
	\delta &=\mbox{ declination of the Earth [$degrees$]}\nonumber\\
	T &=\mbox{ temperature [$^\circ$C]}\nonumber\\
	A &=\mbox{ area covered by solar farm [$m^2$]} \nonumber\\
	\gamma &=\mbox{ temperature coefficient (0.0045)}\nonumber\\
	\tau &=\mbox{ transmittance of the PV module}\nonumber
\end{align}
These values were obtained from the iSEE facts sheet about the UIUC solar farm
\cite{white_solar_2017}. \\

The steam demand is given in units of klbs/hr. APP currently generates steam at
both 850 psi and
325 psi \cite{noauthor_abbott_nodate}.
The steam demand does not distinguish between these two pressures, so we assume
that all of the
steam generated is at 850 psi.
The inlet temperature is $120^\circ$C and exits the boilers at $399^\circ$C
\cite{affiliated_engineers_inc_utilities_2015}.
We can convert the steam flow rate at this pressure to thermal power by

\begin{align}
	P_{th} &= \dot{m}_{steam}c\Delta T \mbox{ [$MW_{th}$]}\\
\intertext{where}
	\dot{m} &= \mbox{ steam flow rate [kg/hr]}\nonumber\\
	c &= \mbox{ specific heat of water [J/kg K]}\nonumber\\
	\Delta T &= \mbox{ the change in temperature [K]}\nonumber
\end{align}

\subsection{Dispatch Model}

The dispatch model is similar to that used by Baker et. al
\cite{baker_optimal_2018}. The net demand is defined by
Eq. \ref{eqn:net-demand}. The dispatch model first checks if renewables satisfy
the campus grid demand.
This can never happen with the current installed renewable capacity at UIUC.
Then a hypothetical micro-reactor will produce electricity to the grid.
Two assumptions will be made: First, the micro-reactor will only produce
electricity and not co-generate like APP; second, the micro-reactor will
operate at 100\% capacity all of the time.
This second assumption is a real possibility given the range of sizes we are
investigating.
If the reactor can meet and exceed the net demand then the dispatch model will
attempt to store the surplus in the campus chilled water system.
If the chilled water tank is full, then the surplus must be sold back to the
grid operator (MISO) and a penalty will be incurred.
When the reactor cannot meet the electric demand, APP will produce electricity
to fill the difference.
UIUC will rely on APP for load following because natural gas boilers can,
currently, ramp more quickly than nuclear reactors.
If either the demand is not covered or all surplus power is not used then a
penalty is incurred.
The penalty is given by \cite{baker_optimal_2018}
\begin{align}
 	p &= \sum_h^{8760} \left(0.4699e^{0.10141\Delta E}\right)\\
 	\intertext{where }\\
 	\Delta E &= \mbox{ $P_{KWh} - netDemand$}\nonumber
\end{align}
The penalty function is applied to a given reactor size and is not reflected in
the LCOE. If a reactor is appropriately sized, it will not incur a penalty.
\begin{figure}[H]
	\centering
	\label{fig:dispatch-model}
	\resizebox{\columnwidth}{!}{%begin resizebox
		\begin{tikzpicture}[node distance = 3cm, auto]
			\node [elli] (start) {Net Demand};
			\node [diam, below of=start] (check1) {Net Demand > 0?};
			\node [rect, rounded corners, right of=check1, xshift=0.5cm] (action1)
			{Reactor produces electricity\\to the grid};
			\node [diam, below of=action1] (check2) {Net demand met?};
			\node [diam, left of=check2, xshift=-2cm] (check3) {Thermal storage
			full?};
			\node [rect, rounded corners, below of=check3] (action2) {Use surplus
			power to fill thermal storage};
			\node [rect, rounded corners, below of=check2, xshift=2cm] (action3)
			{Abbott Power Plant produces electricity to grid};
			\node [rect, rounded corners, right of=action2] (sell) {Sell surplus back
			to MISO};
			\node [diam, below of=action3] (check4) {Remaining demand covered?};
			\node [diam, below of=action2] (check5) {All surplus power used?};
			\node [rect, below of=check4] (penalty) {Pay Penalty};
			\node [diam, right of=check4, xshift=1cm] (hourcheck1) {hour > 8760?};
			\node [diam, left of=check5, xshift=-1cm] (hourcheck2) {hour > 8760?};

			\path[line, line width=0.5mm] (start) -- (check1);
			\path[line, line width=0.5mm] (check1) -- (action1) node [midway] {yes};
			\path[line, line width=0.5mm] (action1) -- (check2);
			\path[line, line width=0.5mm] (check2) -- (check3) node [midway] {yes};
			\path[line, line width=0.5mm] (check2) -- (action3) node [midway] {no};
			\path[line, line width=0.5mm] (check3) -- (sell) node [midway] {yes};
			\path[line, line width=0.5mm] (check3) -- (action2) node [midway] {no};
			\path[line, line width=0.5mm] (action2) -- (check5);
			\path[line, line width=0.5mm] (hourcheck2) |- (start) node [near end]
			{no};
 			\path[line, line width=0.5mm] (check5) -- (hourcheck2) node [midway]
			{yes};
			\path[line, line width=0.5mm] (check5) |- (penalty) node [near start]
			{no};
			\path[line, line width=0.5mm] (check4) -- (penalty) node [midway] {no};
			\path[line, line width=0.5mm] (penalty) -| (hourcheck1);
			\path[line, line width=0.5mm] (hourcheck1) |- (start) node [near end]
			{no};
			\path[line, line width=0.5mm] (action3) -- (check4);
			\path[line, line width=0.5mm] (check4) -- (hourcheck1) node [midway]
			{yes};
			\path[line, line width=0.5mm] (sell) |- (penalty);
			\path[line, line width=0.5mm] (check1) -- (check3) node [midway] {no};
		\end{tikzpicture}
	}% end resizebox
	\caption{Dispatch flow diagram for the UIUC NHES}
\end{figure}

\subsection{Capital Cost}
As in Baker et. al, capital costs are taken in full at year zero. In the first
scenario we examine the simplifying assumption is that there are no capital
costs because the reactor would be provided at no cost to the university. For
the second scenario we will apply the model from Baker et. al. The reference
plant used in that work is an 1100 MWe pressurized water reactor (PWR) which
costs $\$4100$/kWh for an ideal case. $x = 0.64$ to model the economies of
scale \cite{baker_optimal_2018}.

\begin{align}
	CF^{capital} &= \alpha^{year}\left(\frac{P_{\mu}}{P_{ref}}\right)^x \\
	\intertext{where}
		\alpha^{year} &= \mbox{ construction cost at year $0$}\nonumber\\
		P_\mu &= \mbox{ size of the micro-reactor}\nonumber\\
		P_{ref} &= \mbox{ size of the reference plant}\nonumber
\end{align}
